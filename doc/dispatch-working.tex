\documentclass{article}
\usepackage{amsmath}
\usepackage{amssymb}
\title{Specialization Store Working}
\author{Mark Cox}

\newcommand{\true}{\textrm{true}}
\newcommand{\false}{\textrm{false}}

\begin{document}
\maketitle

TODO
\begin{enumerate}
\item Improve the terminology. Precedence, applicability, most
  applicable.
\item Add proofs.
\end{enumerate}

\section{Introduction}

The specialization store system defines a new style of generic
function which can i) dispatch according to optional, keyword or rest
arguments and ii) the operation is determined using argument types
rather than classes.

This document outlines an ordering of specializations for a given set
of input types and that there exists a set of projections which
partition a set of specializations in to disjoint regions.

The operators used in this document are defined as follows
\begin{itemize}
\item $ x \preceq y $ denotes that the set $x$ is a subset of the set $y$.
\item $ x \prec y$ denotes that the set $x$ is a strict subset of the set $y$.
\item $\lnot x $ is logical negation operator.
\item $x \lor y$ is an operator performing the \emph{or} operation.
\item $x \land y$ is an operator performing the \emph{and} operation.
\item $\Pi_{i=1}^n x_i $ is equivalent to $ x_i \land \dots \land x_n $.
\end{itemize}
The standard numerical operators $=$, $<$, $>$, $\leq$ and $\geq$ are
also used.

A symbol $x$ may refer to a tuple containing an arbitrary number of
values, written $x = (x_1, \dots, x_n)$, or may represent a type. The
context of the symbol should make it clear what it represents. Note
that \emph{none} of the above operators are defined for tuples.

The operators defined for tuples are
\begin{itemize}
\item $|x|$ is the number of elements in the typle. e.g. $|x| = n$
  when $ x = ( x_i, \dots, x_n ) $ .
\end{itemize}

\section{Specializations}
There are two different types of specializations defined in
specialization store.
\begin{description}
\item[Fixed arity] These are specializations which can be added to
  store functions which accept a fixed number of required arguments
  and possibly optional arguments and/or keyword
  arguments. Specializations belonging to these functions are
  represented by $ t \equiv (t_1, \dots, t_n) $.
\item[Variable arity] These specialization belong to store functions
  which define a \texttt{\&rest} argument without using
  \texttt{\&key}. These specializations are represented by
  $ x \equiv (t_{1}, \dots, t_{n_x}, \sigma_x, \mu_x) $ where $n_x$
  represents the number of required arguments for the specialization,
  $\sigma_x$ is a non negative integer indicating how many other
  arguments the specialization accepts and $\mu_x$ is the type of each
  of the other $\sigma_x$ arguments. In the context of argument types
  $t$, the value $\sigma_t = |t| - n_x$ is the number of non required
  arguments in the tuple $t$ and $\mu_t$ is the most specific type for
  which each non required argument.
\end{description}

\section{Ordering}
\label{sec:ordering}
\subsection{Fixed arity}
In this section we show how a specialization is selected for an
invocation with the argument types $t \equiv (t_1, \dots, t_n)$.

A specialization is selected by sorting the specializations using a
predicate $h(x,y;t)$ where $x$ and $y$ denote a specialization tuple
and $t$ is the tuple containing the argument types. The function $h$
is true if $x$ is applicable to the argument types $t$ and is a more
suitable specialization than $y$. Its definition is as follows
\begin{align}
  h(x,y;t) &\equiv g(x;t) \land (\lnot g(y;t) \lor f(x,y) )
\end{align}
where $g$ is the applicability function
\begin{align}
  g(x;t) &\equiv \Pi_{i=1}^n (t_i \preceq x_i)
\end{align}
and $f$ is the precedence function
\begin{align}
  f(x,y) &\equiv \Pi_{i=1}^n (x_i \preceq y_i)
\end{align}

\subsection{Variable arity}
The functions $g$, $f$ and $h$ for specializations with variable arity
are
\begin{align}
  g(x;t) &\equiv (n_x \leq |t| \leq n_x + \sigma_x)
                 \land \Pi_{i=1}^{n_x} (t_i \preceq x_i)
                 \land ( \mu_t \preceq \mu_x ) \\
  f(x,y) &\equiv (n_x \geq n_y)
                 \land \Pi_{i=1}^{n_y}(x_i \preceq y_i)
                 \land ( \mu_x \preceq \mu_y )\\
  h(x,y;t) &\equiv g(x;t) \land (\lnot g(y;t) \lor f(x,y) )
\end{align}

\section{Dispatch Tree}
An applicability function $g$ and a discrimination function $f$
provide a method of selecting a specialization from a set $Z$ which
has the highest precedence for a set of arguments. This method of
selection is not very efficient as it requires sorting the set of
specializations.

In this section we seek to find a function $d : t \rightarrow x$ which
computes a specialization $x \in Z$ which has the highest precedence
for input arguments of type $t$.

The key property of the function $d$ is that its computational
complexity is no greater than the $\max_i |x_i|$ where $x_i \in Z$.

\subsection{Criteria}
The following cases need to proven in order to show that the function
$d$ is equivalent to the sorting method.
\begin{align}
  \begin{cases}
    g(d(t); t) \equiv \true & h(d(t),y;t) \equiv \true \;\; \forall y \in Z \\
    g(d(t); t) \equiv \false & g(y;t) \equiv \false \;\; \forall y \in Z
  \end{cases}
\end{align}

\noindent The sorting function $h(x,y;t)$ is defined as
\begin{align}
  h(x,y;t) &\equiv g(x;t) \land (\lnot g(y;t) \lor f(x,y) )
\end{align}
which has the following truth table

\begin{table}[h]
\centering
\begin{tabular}{|c|c|c||c|}
\hline
$g(x;t)$ & $g(y;t)$ & $f(x,y)$ & $h(x,y;t)$ \\
\hline
F & F & F & F \\
F & F & T & F \\
F & T & F & F \\
F & T & T & F \\
T & F & F & T \\
T & F & T & T \\
T & T & F & F \\
T & T & T & T \\
\hline
\end{tabular}
\end{table}

The precedence function for the fixed arity and variable arity case
both share the term
\begin{align}
  \Pi_{i=1}^{\min(n_x,n_y)} x_i \preceq y_i
\end{align}
which, according to the function $h$, requires an input argument type
$t_i$ to satisfy
\begin{align}
  (t_i \preceq x_i) \land [\lnot (t_i \preceq y_i) \lor (x_i \preceq y_i)].
\end{align}

\subsection{Fixed Arity}
Assume that we have a set of specializations $Z$ for an arity $n$
store function which are partially applicable.
\begin{align}
  x_1 & \equiv (x_{11}, \dots, x_{1n}) \nonumber \\
  x_2 & \equiv (x_{21}, \dots, x_{2n}) \nonumber \\
  \vdots & \vdots                     \nonumber \\
  x_k & \equiv (x_{k1}, \dots, x_{kn}) \nonumber
\end{align}
It is assumed that no two specializations are the same.

Partial applicability is defined by the set $P$ which represents
knowledge about any of the input argument types $t_j$. Specifically
\begin{align}
  (p_j \in P) \rightarrow (t_j \preceq p_j)
.
\end{align}

A specialization $x_i$ is said to be partially applicable if
\begin{align}
  \exists_j \left[(p_j \preceq x_{ij}) \land (p_j \in P) \right]
.
\end{align}

A specialization $x_i$ is said to be applicable if
\begin{align}
  \forall_j \left[(p_j \preceq x_{ij}) \land (p_j \in P) \right]
\end{align}

The function $d$ created for the fixed arity case is a decision tree
which involves selecting a type $p_j$ to use as a test for a specific
input argument type $t_j$.

The type $p_j$ is selected from the $x_{ij}$ such that
\begin{enumerate}
\item $\exists_i (p_j \equiv x_{ij})$
\item $\{ x_{ij} | x_{ij} \prec p_j \} \equiv \emptyset$
\item $p_j \notin P$
\end{enumerate}

The type $p_j$ is then used to partition the specializations $Z$ in to
a set $X$ and $Y$ such that
\begin{align}
  X & \equiv \{ x_i | p_j \preceq x_{ij} \} & Y & \equiv (Z - X) \cup \{ x_i | p_j \prec x_{ij} \}
\end{align}

This process continues by splitting the set $X$ with
$P = P \cup \{ p_j \}$ and splitting the set $Y$ with $P =
P$. Splitting stops when no $p_j$ can be obtained.

There may be more than one specialization in $X$ when $P$ is full
e.g. the tuples $x_1 = (\alpha)$ and $x_2 = (\alpha \lor \beta)$. This
can be resolved by sorting the specializations using $h()$.

\subsection{Variable Arity}
The tree building process for the variable arity case proceeds by
partitioning the specializations using the first $\min_i n_{x_i}$
arguments as if it were a fixed arity problem.

A leaf in the resulting tree will contain a set of specializations
$Z$. These specializations can be split in to two categories, those
which can be invoked with $c$ arguments and those that can be invoked
with more than $c$
\begin{align}
  X & = \{ x_i | n_{x_i} \leq c \leq n_{x_i} + \sigma_{x_i} \} & Y &= \{ x_i | c < n_{x_i} + \sigma_{x_i} \}
\end{align}
where $c$ is defined as
\begin{align}
  c = \min_i n_{x_i} \textrm{ s.t. } x_i \in Z
\end{align}

The set $X$ represents a fixed arity problem and thus can be
partitioned using the strategy outlined in the previous section. The
above process is recursively applied to set $Y$.

\end{document}

%%% Local Variables:
%%% mode: LaTeX
%%% mode: TeX-PDF
%%% End:
